\chapter{Discussion}
\label{chap:discussion}

In this chapter the advantages and usefulness of the interactive map as well as the user-friendliness of this interactive system are highlighted. Furthermore, the weaknesses and difficulties of the system are addressed based on the observation and feedback from the online survey studies. Finally, some complementary approaches to resolve some of these difficulties are outlined.

One of the major advantages of this interactive map is its ability to enable users to get a quick overview of the geographical location of German power plants and high voltage power transmission lines inside Germany.  The interactive map also provides technical data of all power plants and power lines respectively inside the pop-up window and label. The map also provides two more brushing and linking features inside the pop-up window. With “Go to Energy Charts” function, it is possible to see the hourly production of a singular power plant on the energy chart in the form of stacked area chart. “Compare” function generates a comparison table inside the webpage and selected data of that power plants are listed inside the table. Whereas, from the comparison table it is also possible to see the hourly production of the power plant group listed in the table. The navigation menu allows user to select different energy sources and display them on the map in groups. One can select German states from a drop down list to have a closer look and get an idea about the density of power plants and power lines within that specific region. With this new interactive map in combination with Energy Charts, users can get information on the contribution of renewable and non-renewable energy to power generation in Germany. Survey participants were enthusiastic about having these features in one place. They mentioned that the tool is very informative and easy to use. Although, the map fails to provides some information that user wished for. The data for power plants are periodically collected from European Energy Exchange (EEX) in Leipzig which include power plants which are capable of producing an output of 100MW or more. Our interactive map provides no information on solar power plants and covers not all energy sources with lower generation capacity (less than 100 MW) due to a voluntary reporting. During the survey, participants were asking for the information on solar and missing medium sized power plants. The interactive map also shows the high voltage power transmission lines, namely 110 kV, 220 kV and 380 kV. Participants also mentioned their interest in power lines and asked for more technical data about power lines. In the survey, they also mentioned high loading time of power lines on the map. The reason of high loading times was forecasted while developing the interactive map. The large GeoJSON files cause high loading times and for this purpose irresponsiveness is visible on the map. A dedicated database could solve this problem. Map API provides a fully clickable interactive system for the user. Therefore, many users found it user-friendly. Although, confusion has been raised for novice users while they saw the initial cluster view and didn’t understand how to proceed. They requested to remove this feature and didn’t find it useful. Little experienced users who have been using interactive maps or similar applications found it easy to use. Participants found the markers, navigation menu, and user interface elements self-explanatory. Nevertheless, the interactivity does not eliminate the need for a tutorial by now. One of the participants could not find the comparison table while taking part in the survey. One approach to solve this problem could be a legend mentioning the map features and its operation on the page. Interactive data drill down and brushing-and-linking function make a strong connection with Energy Charts where users can get up-to-date data on German electricity production. 

The interactive map of the Energy charts is consistently expanding therefore the view of the map user interface and its functionality might be changed over time. At the same time, the operation and other requested additional features by participants will be considered for increasing its usability from novice to expert users. The actual goal of the Energy Charts is to provide a solid data base and even more to make the energy data and German \textit{"Energiewende"} transparent to the interested group of users.  

