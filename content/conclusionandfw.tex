\chapter{Conclusion and Future Work}
\label{chap:conclusion}

In this thesis, an interactive map is developed for visualizing the geographical location of German power plants as well as providing relevant information about the plants, such as, source category, nominal power, and owner. Furthermore, the map also visualizes the distribution of high voltage power lines and their connectivity to the power plants. This new feature has been added as an extension to the Energy Charts – a portal for visualizing the energy data which is driven by Fraunhofer ISE. Energy Charts present interactive charts on electricity production, electricity stock market prices and import/export of electricity which is very popular and widely used by people from different countries. In the frame of this work, the new interactive application has been integrated to extend a new way for accessing electricity production data for each power plant. Glyph-based visualization technique is used to design the location marker according to the semantics of the power plants. Every time a user clicks on the power plant, a pop-up window appears with technical information on the selected power plant, for example, source category, nominal electricity generation capacity and owner. The visualization tool also offers brushing and linking functionality with energy charts. The map and the charts for the hourly electricity production chart are interlinked in both ways. The map offers two different ways for establishing this connection with energy charts. First, map users can see the hourly production of a singular power plant or group of power plants on the Energy Charts. Second, with the function \textit{"Compare"}, the selected power plants are listed in a table for comparison and users can see their production on the Energy Charts. Interactive map source selection includes hydro, biomass, nuclear, brown coal (lignite), hard coal, oil, gas, pumped storage, seasonal storage, wind, and waste power plants. One can either see the power plants in groups or select the plants according to their groups. No data on solar power plant are available in this information visualization since the EEX does not contain them. Furthermore, Energy Charts is also connected with the interactive map. The users can see the location of all power plants on the map if a source category is selected on the energy charts. Therefore, both map and Energy Charts are interlinked with each other.  

The online survey evaluation study showed that the participants adopted the tool very well and talked very positive about the new application. Furthermore, participants found the tool very informative and were using this tool for getting information every day which is noticed from the online access pattern of the user all over the world.

\section*{Future Work}

In the following, new design concepts and some must-have features which came up during the development of the interactive map and from the ones which were provided as feedback by the online survey participants, will be presented. 

\textbf{Database}
\newline
Implementing a database will be a key part in the further development of the system. It would enable adding new features to the map which are computationally expensive with the existing framework. At the moment, Energy Charts and interactive map are parsing JSON files on the client side and doing large calculations in the browser which is not efficient. The performance of the map decreases while parsing large GeoJSON files. A database management system would ease tasks by performing these large calculations in the server-side.

\textbf{Marker cluster}
\newline
According to the EEX data, there are many power plants located in the same geographical location. After plotting them on the map using markers, users can only see a single marker on that position. Other plants are hidden by that single marker. The marker cluster plugin solved this problem to some extent. However, users were confused by this visualization technique. Therefore, a new way of visualizing this cluster would help for giving the user more detailed information about their existence on the map.

\textbf{Compare different sources}
\newline
Many participants wished for comparing multiple power sources in the comparison table as well as in the Energy Charts. This feature would enable users to compare electricity production of different power source units. Therefore, a solution needs to be implemented to merge several JSON files of different sources and provide this new filtered file as compatible input in the NVD3 stacked area chart. 

\textbf{Information on solar plants}
\newline
During the survey participants responded that they were very much interested in having information on solar power plants. This interactive data visualization tool would be more complete and informative if the map can provide information on solar power plants. 

