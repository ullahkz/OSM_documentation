\chapter{Conclusion and Future Work}
\label{chap:conclusion}

The online visualization tool of German power plants is a fully clickable interactive map providing a quick overview of the location of all power plants listed on the European Energy Exchange (EEX) in Leipzig. In addition, the map also illustrates the high voltage power transmission lines (110 kV, 220 kV and 380 kV) running through all Germany. This interactive visualization map has been added to the Energy Charts, driven by the Fraunhofer Institute for Solar Energy System (ISE). Energy charts presents interactive charts on electricity production, electricity stock market prices and import/export of electricity which is very popular and widely used by people from different countries. In the frame of this work, the new interactive application has been developed to extend a new way for accessing electricity production data for each power plant. Glyph based visualization technique is used to design the location marker according to the semantics of the power plants. Every time user clicks on the power plant, a pop-up window appears with technical information about the power plants for example, source category, nominal electricity generation capacity and owner. The visualization tool also offers brushing and linking functionality with energy charts. The map and hourly electricity production chart are interlinked in both ways. The map offers two different ways for establishing this connection with energy charts. Firstly, map users can see the hourly production of singular power plant or group of power plants on the energy chart. Secondly with the function “Compare”, the selected power plants are listed in a table for comparison and at the end user can see their production on energy charts. Interactive map source selection includes hydro, biomass, nuclear, brown coal (lignite), hard coal, oil, gas, pumped storage, seasonal storage, wind, and garbage power plants. One can either see the power plants in groups or select the plants according to their groups. No data on solar power plant are available in this information visualization, since the EEX list contains only those plants that are larger than 100 MW. There are no solar plant exists in Germany with such a large capacity.  However, energy chart provides hourly electricity production of hydro, nuclear, lignite, hard coal, gas, oil, pumped storage, and wind power plants units. Furthermore, energy chart is also connected with the interactive map. User can see the location of all power plants on the map if a source category is selected on the energy charts. Therefore, both map and energy charts are interlinked with each other. 

The online survey evaluation study showed that the participants adapted the tool very well and very positive about the new application. Furthermore, participants found the tool very informative and using this tool for getting everyday which is noticed from the online access pattern of the user all over the world.  

\section*{Future Work}

In the following, new design concepts and some must-have features which arises during the development of the interactive map and from the feedback of online survey participants will be presented. 

\textbf{Database}
\newline
Implementing a database will be a key part in the further development of the system. It would enable adding new features to the map which are computationally expensive with existing framework. At the moment, energy charts and interactive map are parsing JSON files in the client side and doing large calculations in the browser which is not efficient. The performance of the map affects while parsing large GeoJSON file. A database management system would ease tasks by performing this large calculations in the server-side.

\textbf{Marker Cluster}
\newline
According to the EEX data, there are many power plants located in the same geographical location. After plotting them on the map using markers, users can only see a single marker on that position. Other plants are hidden by that single marker. The marker cluster plugin solved this problem to some extent. However, users were confused by this visualization technique. Therefore, a new way of visualizing this cluster would help for giving user more detailed information about their existence the map.

\textbf{Compare different sources}
\newline
Many participants wished for comparing multiple power sources in the comparison table as well as in the energy charts. This feature would enable users to compare electricity production of different power source units.

\textbf{Information on Solar plants}
\newline
During the survey it was clear that participants are very much interested in having information on solar power plants. This interactive data visualization tool would be more informative if the map can provide information on solar PV and solar thermal power plants. 

