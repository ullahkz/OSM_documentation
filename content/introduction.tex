\chapter{Introduction}

Data visualization allow us to take valuable information from itself instead of being bored from reading rows upon rows of spreadsheet data. It also allows extracting complex data into graphic visual representations and to communicate information clearly and efficiently via interactive graphical presentations. This visual communication process becomes more efficient when interactivity is added to it. Since July 2014, the Fraunhofer Institute for Solar Energy Systems (ISE) has been providing interactive charts on electricity production, electricity stock market prices and import/export of electricity in Germany. This charts became very popular and were widely used by people from different profession namely scientists, politicians, journalists as well as online/printed media. Due to the high popularity and continuous development of these interactive charts, an additional interactive map is developed in this Masters Thesis to make the whole framework more informative and interesting for the users. The aim of this visualization tool is to develop an interactive map to illustrate the geographical location of power plants in Germany with OpenStreetMap technology. On top, this data layer is providing necessary information about power plants, source, capacity, operator, start date and other additional information which a user might be interested in. Furthermore, this new visualization framework is interconnected with the existing charts of electricity production per unit which enhances the framework even more by an additional layer providing more visibility and information on the selected power plants(on the map).

The data is periodically collected from European Energy Exchange(EEX) in Leipzig, which include power plants that are larger than 100MW. Their reporting is obligatory to the EEX in Germany. In addition, the data also contains information on voluntary reporting companies which are generating lower power output. However, the data used for this visualization does not include all renewable and non-renewable power plants. Apart from this, this interactive map also illustrates the power lines of Germany, specifically the 110KV, 220KV and 380KV lines. The data set for these power lines is extracted using Overpass turbo API. 

Due to the lack of frameworks that supply a sufficient visualization for this kind of scenario and data, it seemed to be essential to develop additional interacting factors that can be added to present the data in a way which tells the story behind the data more clearly. On the other hand, there is a scope of giving the user extra control to make the interactive chart more useful. The main focus of this project is to visualize the data overlaid on a geographical projection using open source data mapping tool and deploy this framework to the web environment. In addition, a fully interactive data visualization is provided by adding custom API and glyph-base visualization technique, which allows technical and non-technical experts easy to grips with the context.
 
\section*{Outline}
This structure of the thesis as follows:
\begin{description}

\item[Chapter~\ref{chap:relatedWork&Foundation} -- \nameref{chap:relatedWork&Foundation}:]
In this chapter, related work in the field of interactive maps as well as time series chats which have a focus on energy data visualization are discussed. 

\item[Chapter~\ref{chap:background} -- \nameref{chap:background}:]
In this chapter, a brief overview on this visualization tool, used libraries, data file format and working environments is given.

\item[Chapter~\ref{chap:softwareSystem} -- \nameref{chap:softwareSystem}:] In this chapter, requirement analysis of the tool is discussed as well as the architecture and user interface are described. Furthermore, we explained different interactive functions that map offers and also a brief discussion on power line visualization techniques are optimized for better performance. 
 
\item[Chapter~\ref{chap:evaluation} -- \nameref{chap:evaluation}:] In this chapter, online survey evaluation result of the interactive map and user feedback is discussed.   

\item[Chapter~\ref{chap:discussion} -- \nameref{chap:discussion}:] In this chapter, the advantage and weakness of the visualization tool are discussed.

\item[Chapter~\ref{chap:conclusion} -- \nameref{chap:conclusion}:] In this chapter, an overview of the whole work, summary of the online evaluation result and scope of future work is discussed .

\end{description}

\section*{Acronym}
In this paper several abbreviations are used which may not be commonly known to everyone. Therefore, the following list is added to clarify the used terms and to avoid confusions with similar abbreviations. 

\begin{table}[H]
\centering
%\caption{List of abbreviations used in this thesis}
\label{Acronym}
\begin{tabular}{l|llll}
\textbf{Acronym} & \textbf{Definition} &  \\ \hline
OSM         & Open Street Map & \\
UI         & User Interface &  \\
JSON		& JavaScript Object Notation & \\  
API		& Application Programming Interface & \\
CDN		& Content Delivery Network & \\  
\end{tabular}
\end{table}

