%Die Angabe des schlauen Spruchs auf diesem Wege funtioniert nur,
%wenn keine Änderung des Kapitels mittels den in preambel/chapterheads.tex
%vorgeschlagenen Möglichkeiten durchgeführt wurde.

\chapter{Related Work}
\label{chap:k2}
%\vspace{-3cm}
%\vspace{2cm}

%\section{Quellcode}
Listing~\ref{lst:ListingANDlstlisting} zeigt, wie man Programmlistings einbindet.  Mittels \texttt{\textbackslash lstinputlisting} kann man den Inhalt direkt aus Dateien lesen.

%Listing-Umgebung wurde durch \newfloat{Listing} definiert
\begin{Listing}
\begin{lstlisting}
<listing name="second sample">
  <content>not interesting</content>
</listing>
\end{lstlisting}
\caption{lstlisting in einer Listings-Umgebung, damit das Listing durch Balken abgetrennt ist}
\label{lst:ListingANDlstlisting}
\end{Listing}

Quellcode im \lstinline|<listing />| ist auch möglich.

%\section{Abbildungen}
Die Abbildungen~\ref{fig:chor1} und~\ref{fig:chor2} sind für das Verständnis dieses Dokuments
wichtig. Im Anhang zeigt Abbildung~\vref{fig:AnhangsChor} erneut die komplette Choreographie.

%Die Parameter in eckigen Klammern sind optionale Parameter - z.B. [htb!]
%htb! bedeutet: "Liebes LaTeX, bitte platziere diese Abbildung zuerst hier ("_h_ere"). Falls das nicht funktioniert, dann bitte oben auf der Seite ("_t_op"). Und falls das nicht geht, bitte unten auf der Seite ("_b_ottom"). Und bitte, bitte bevorzuge hier und oben, auch wenn's net so optimal aussieht ("!")
%Diese sollten nach Möglichkeit NICHT verwendet werden. LaTeX's Algorithmus für das Platzieren der Gleitumgebung ist schon sehr gut!
\begin{figure}
  \begin{center}
    \includegraphics[width=\textwidth]{choreography.pdf}
    \caption{Beispiel-Choreographie}
    \label{fig:chor1}
  \end{center}
\end{figure}

\begin{figure}
  \begin{center}
    \includegraphics[width=.8\textwidth]{choreography.pdf}
    \caption[Beispiel-Choreographie]{Die Beispiel-Choreographie. Nun etwas kleiner, damit \texttt{\textbackslash textwidth} demonstriert wird. Und auch die Verwendung von alternativen Bildunterschriften für das Verzeichnis der Abbildungen. Letzteres ist allerdings nur Bedingt zu empfehlen, denn wer liesst schon so viel Text unter einem Bild? Oder ist es einfach nur Stilsache?}
    \label{fig:chor2}
  \end{center}
\end{figure}

Das SVG in \cref{fig:directSVG} ist direkt eingebunden, während der Text im SVG in \cref{fig:latexSVG} mittels pdflatex gesetzt ist.
\todo{Falls man die Graphiken sehen möchte, muss inkscape im PATH sein und im Tex-Quelltext \texttt{\textbackslash{}iffalse} und \texttt{\textbackslash{}iftrue} auskommentiert sein.}

\iffalse % <-- Das hier wegnehmen, falls inkscape im Pfad ist
\begin{figure}
\centering
\includegraphics{svgexample.svg}
\caption{SVG direkt eingebunden}
\label{fig:directSVG}
\end{figure}

\begin{figure}
\centering
\def\svgwidth{.4\textwidth}
\includesvg{svgexample}
\caption{Text im SVG mittels \LaTeX{} gesetzt}
\label{fig:latexSVG}
\end{figure}
\fi % <-- Das hier wegnehmen, falls inkscape im Pfad ist

%\section{Tabellen}
Tabelle~\ref{tab:Ergebnisse} zeigt Ergebnisse.
\begin{table}
  \begin{center}
    \begin{tabular}{ccc}
	\toprule
	\multicolumn{2}{c}{\textbf{zusammengefasst}} & \textbf{Titel} \\ \midrule
	Tabelle & wie & in \\
	\url{tabsatz.pdf}& empfohlen & gesetzt\\
	
	\multirow{2}{*}{Beispiel} & \multicolumn{2}{c}{ein schönes Beispiel}\\
	 & \multicolumn{2}{c}{für die Verwendung von \enquote{multirow}}\\
	\bottomrule
    \end{tabular}
    \caption[Beispieltabelle]{Beispieltabelle -- siehe \url{http://www.ctan.org/tex-archive/info/german/tabsatz/}}
    \label{tab:Ergebnisse}
  \end{center}
\end{table}
