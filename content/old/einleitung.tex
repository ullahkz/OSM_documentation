\chapter{Introduction}
Interactive visual installations have already become part of our daily life. Public displays are informing us about the weather and news, projector installations are an integral part of many art exhibitions, and media facades overlay whole buildings with appealing content. Currently this content is mostly prepared on desktop computers and then remotely deployed. However, this often does not provide the expected results; a common example is readability of text. Although text prepared on a personal computer may appear well readable on this screen, it may be completely unreadable in the target context due to many factors such as screen resolution, density, projection surface, viewing angle, or just a too large distance from the target audience to the screen. We believe that the effect of visual content can be best assessed and adjusted onsite and in context of the target display.

\section*{Outline}
Die Arbeit ist in folgender Weise gegliedert:
\begin{description}
\item[Kapitel~\ref{chap:k2} -- \nameref{chap:k2}:] Hier werden werden die Grundlagen dieser Arbeit beschrieben.
\item[Kapitel~\ref{chap:zusfas} -- \nameref{chap:zusfas}] fasst die Ergebnisse der Arbeit zusammen und stellt Anknüpfungspunkte vor.
\end{description}
